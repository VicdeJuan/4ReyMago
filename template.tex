\documentclass[11pt]{article}

\usepackage{xspace}
\usepackage[utf8]{inputenc}
\usepackage{aurical}
\usepackage{pbsi}
\usepackage{lettrine}
\usepackage{soulutf8}

\setlength{\DefaultNindent}{1em}
\setcounter{DefaultLines}{1}
\input Romantik.fd
\newcommand*\initfamily{\usefont{U}{Romantik}{xl}{n}}
\renewcommand{\LettrineFontHook}{\initfamily}


\usepackage[margin=1in]{geometry}


\DeclareRobustCommand{\augiefamily}{%
  \fontfamily{augie}\fontseries{m}\fontshape{n}\selectfont}
\DeclareTextFontCommand{\textaugie}{\augiefamily}


\renewcommand{\baselinestretch}{1.2}
\setlength{\parindent}{4em}
\setlength{\parskip}{1.5em}


\newcommand{\receiver}{Regalado\xspace}
\newcommand{\giver}{Regalador\xspace}

\begin{document}

%\Fontskrivan
\bsifamily
%\usefont{T1}{pbsi}{xl}{n}}

%\twistshape

¡Hola \giver!
Tengo la buena noticia de informarte que esta Navidad pasas a formar parte del equipo de los Reyes Magos, como cuarto Rey Mago. \ul{¡Enhorabuena!}
Verás, \LARGE{\receiver} \large necesita un regalo que de alguna forma pueda acercarle un poquito más a Dios.
%
Nosotros, Melchor, Gaspar y Baltasar, vamos a cubrir los regalos que haya pedido a su familia, algún amigo... pero a tanto no llegamos. Además, tú conoces a \receiver mejor que nosotros. ¡Seguro que lo haces mejor!
Contamos contigo para que le hagas un regalo a \receiver y nos gustaría que tuviera las siguientes características:
\vspace{-1em}
\begin{itemize}
\item Un regalo hecho con cariño porque \ul{es alguien con quien compartes tu fe y tu vida}. Que le ayude a darse cuenta del regalo que nos hace Dios al tener este grupo de catequesis. ¡Hay infinidad de maneras de demostrarlo! Los tres sabios confiamos en tu creatividad y originalidad.
\item Nos gustaría que fuera un \ul{regalo cuidado, hecho a mano}, hecho por ti. Es una buena forma de demostrarle la suerte que tiene al compartir este camino contigo.
\item Si necesitaras comprar algo de material, nos gustaría que \ul{no te gastes más de 5 euros} pues es mejor algo sencillo y cuidado antes que un regalo complicado y costoso.
\item Finalmente es importante que sepas que tendrás que traer el regalo el \LARGE{\ul{viernes 12 de enero a tu grupo.}}\large
\end{itemize}
Firmado:
\begin{center}
\begin{tabular}{ccc}
\augiefamily Melchor ,& \augiefamily Gaspar & \augiefamily \& Baltasar\\\hline
\end{tabular}
\end{center}


\newpage


\Fontauri\bfseries\slshape


\vspace*{\fill}
\begin{center}
{\fontsize{65}{130}\textsc{\giver}}
\end{center}
\vspace*{\fill}


\end{document}